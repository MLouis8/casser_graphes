\documentclass[aspectratio=169]{beamer}
\usetheme{metropolis}

\usepackage{roboto}
\usepackage{mathtools}
\usepackage{fixmath}
\usepackage{graphicx}
\usepackage{tikz}
\usepackage{stmaryrd}
\usepackage{svg}
\usepackage{pdfpages}
\usepackage{algpseudocode}
\usepackage{algorithm}

\graphicspath{ {../images/} }
\setbeamertemplate{navigation symbols}{}
\setbeamercolor{background canvas}{bg=}

\newcommand{\N}{\mathbb{N}}
\DeclarePairedDelimiter\Brackets{\llbracket}{\rrbracket}

\title{Classification des coupes, premier regard sur la betweenness}
\subtitle{Stage Casser des Graphes}
\author{Louis Milhaud}
\institute{Complex Networks - LIP6}
\date{\today}

\begin{document}

    \AtBeginSection[]{
    \begin{frame}
    \vfill
    \centering
    \begin{beamercolorbox}[sep=8pt,center,shadow=true,rounded=true]{title}
        \usebeamerfont{title}\insertsectionhead\par%
    \end{beamercolorbox}
    \vfill
    \end{frame}
    }
    % frame 1
    \begin{frame}
        \titlepage
    \end{frame}

    % frame 2
    \begin{frame}
        \frametitle{Outline}
        \tableofcontents
    \end{frame}

    % frame 3
    \section{Classification des coupes}

    \subsection{analyse des fréquences}
    \begin{frame}{20 arêtes les plus coupées avec $\epsilon = 0.01$}
        \includegraphics{20_most_frequent001.pdf}
    \end{frame}

    \begin{frame}{20 arêtes les plus coupées avec $\epsilon = 0.03$}
        \includegraphics{20_most_frequent003.pdf}
    \end{frame}

    \begin{frame}{20 arêtes les plus coupées avec $\epsilon = 0.1$}
        \includegraphics{20_most_frequent01.pdf}
    \end{frame}

    \subsection{Classification: critères}
    \begin{frame}{critère d'intersection}
        Soit $c_1$ et $c_2$ deux coupes, $c_1 \sim_\epsilon c_2$ lorsque:
        $$|c_1 \cap c_2| \geq \epsilon \cdot |c_1|\quad\epsilon\in [0, 1]$$
    \end{frame}
    \begin{frame}{critère de voisinage}
        Soit $c_1$ et $c_2$ deux coupes, $c_1 \sim_k c_2$ lorsque:
        $$\forall e_1\in c_1\ \exists e_2\in c_2,\ d(e_1, e_2)\leq k\quad k\in\N$$
    \end{frame}
    \begin{frame}
        \frametitle{ce qui a été testé}
        \begin{itemize}
            \item intersection fonctionne rapidement, problème de précision
            \item voisinage trop lent (python)
            \item critères mixes toujours trop lent et peu d'améliorations
        \end{itemize}
    \end{frame}

    \subsection{Classification: méthode}
    \begin{frame}{méthode du représentant}
        Une coupe arbitraire est la représentante d'une classe, le critère est appliqué par rapport a elle.\\
        \vspace*{5pt}
        \emph{Avantages:}
        \begin{itemize}
            \item classe réelle (faite à partir de caractéristiques observées)
            \item simple a appliqué
        \end{itemize}
        \emph{Inconvénient:}
        \begin{itemize}
            \item on ne prend pas le meilleur représentant
            \item trop restrictif ?
        \end{itemize}
    \end{frame}

    \begin{frame}
        \frametitle{méthode du représentant avec critère d'instersection, choix du $\epsilon$}
        \includegraphics[scale=0.42]{inter_rpz_plots_01.pdf}
        \includegraphics[scale=0.42]{inter_rpz_plots_001.pdf}
    \end{frame}

    \begin{frame}
        \frametitle{méthode du représentant avec critère d'instersection, choix du $\epsilon$}
        \begin{center}
            \includegraphics[scale=0.6]{inter_rpz_plots_003.pdf}
        \end{center}
    \end{frame}
    \begin{frame}
        \frametitle{méthode du représentant avec critère d'intersection, visualisation}
    
        \includegraphics[scale=0.75]{visual_rpz_inter05.pdf}
    
    \end{frame}
    \begin{frame}{méthodes prédéfinies}
        On défini à l'avance les classes, on classe en fonction des critères prédéfinis.\\
        \emph{Avantages:}
        \begin{itemize}
            \item On paramétrize le niveau de contrainte et précision des classes
        \end{itemize}
        \emph{Inconvénients:}
        \begin{itemize}
            \item On peut masquer / oublier une classe
        \end{itemize}

        Pas essayé car besoin de données géométriques (pour l'instant tout est fait sur le graphe KaHIP)
    \end{frame}

    \begin{frame}{méthode du meilleur représentant}
        Méthode du représentant mais on itère récursivement dans la classe pour prendre le meilleur représentant.\\
        \vspace*{5pt}
        \emph{Avantage:}
        \begin{itemize}
            \item Prend un bon représentant
            \item n'oublie aucun éléments
        \end{itemize}
        \emph{Inconvénients:}
        \begin{itemize}
            \item Unique classe ou aucune classe (inutile)
        \end{itemize}
    \end{frame}

    \begin{frame}
        \frametitle{méthode de division itérative}
        On choisi $k$ représentants différents d'au moins $t$.\\
        On classe les éléments restants par rapport à leur proximité avec le représentant
        \begin{itemize}
            \item[+] rapide $O(2kn)$ (linéaire)
            \item[+] toutes les coupes sont classées
            \item[+] les représentants sont forcéments différents 
            \item[-] peut être pas les meilleurs représentants 
            \item[-] classé à la proximité (le meilleur n'est pas forcément bon) 
        \end{itemize}
    \end{frame} 

    \begin{frame}
        \frametitle{Résultats avec proximité intersection, 4 classes}
    
        \includegraphics[scale=0.7]{visual_iterative9-08.pdf}
    
    \end{frame}

    
    % \begin{frame}
    %     \frametitle{Conclusion sur la classification}
    %     \begin{itemize}
    %         \item intersection fonctionne car pas de chemins parallèles
    %         \item on observe plusieurs types de coupes, mais une revient plus souvent
    %     \end{itemize}
    % \end{frame}
    \section{Betweenness vs fréquence}

    \begin{frame}{Corrélation des deux mesures}
    \begin{itemize}
        \item $\epsilon = 0.01$: Correlation: $0.1$
        \item $\epsilon = 0.03$: Correlation: $0.06$
        \item $\epsilon = 0.1$:  Correlation: $0.03$
    \end{itemize}
    \end{frame}

\end{document}

