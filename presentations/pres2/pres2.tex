\documentclass[aspectratio=169]{beamer}
\usetheme{metropolis}

\usepackage{roboto}
\usepackage{mathtools}
\usepackage{fixmath}
\usepackage{graphicx}
\usepackage{tikz}
\usepackage{stmaryrd}
\usepackage{svg}
\usepackage{pdfpages}

\graphicspath{ {./images/} }
\setbeamertemplate{navigation symbols}{}
\setbeamercolor{background canvas}{bg=}

\newcommand{\N}{\mathbb{N}}
\DeclarePairedDelimiter\Brackets{\llbracket}{\rrbracket}

\title{Classification des coupes, premier regard sur la betweenness}
\subtitle{Stage Casser des Graphes}
\author{Louis Milhaud}
\institute{Complex Networks - LIP6}
\date{\today}

\begin{document}

    \AtBeginSection[]{
    \begin{frame}
    \vfill
    \centering
    \begin{beamercolorbox}[sep=8pt,center,shadow=true,rounded=true]{title}
        \usebeamerfont{title}\insertsectionhead\par%
    \end{beamercolorbox}
    \vfill
    \end{frame}
    }
    % frame 1
    \begin{frame}
        \titlepage
    \end{frame}

    % frame 2
    \begin{frame}
        \frametitle{Outline}
        \tableofcontents
    \end{frame}

    % frame 3
    \section{Classification des coupes}

    \subsection{analyse des fréquences}
    \begin{frame}{20 arêtes les plus coupées avec $\epsilon = 0.01$}
        \includepdf[pages=-]{../images/20_most_frequent001.pdf}
    \end{frame}

    \begin{frame}{20 arêtes les plus coupées avec $\epsilon = 0.03$}
        \includepdf[pages=-]{../images/20_most_frequent003.pdf}
    \end{frame}

    \begin{frame}{20 arêtes les plus coupées avec $\epsilon = 0.1$}
        \includepdf[pages=-]{../images/20_most_frequent01.pdf}
    \end{frame}

    \subsection{Classification: critères}
    \begin{frame}{critère d'intersection}
        Soit $c_1$ et $c_2$ deux coupes, $c_1 \sim_\epsilon c_2$ lorsque:
        $$|c_1 \cap c_2| \geq \epsilon \cdot |c_1|\quad\epsilon\in [0, 1]$$
    \end{frame}
    \begin{frame}{critère de voisinage}
        Soit $c_1$ et $c_2$ deux coupes, $c_1 \sim_k c_2$ lorsque:
        $$\forall e_1\in c_1\ \exists e_2\in c_2,\ d(e_1, e_2)\leq k\quad k\in\N$$
    \end{frame}

    \subsection{Classification: méthode}
    \begin{frame}{méthode du représentant}
        Une coupe tirée au hasard est la représentante d'une classe, le critère est appliqué par rapport a elle.\\
        \emph{Avantages:}
        \begin{itemize}
            \item classe réelle (faite à partir de caractéristiques observées)
            \item simple a appliqué
        \end{itemize}
        \emph{Inconvénient:}
        \begin{itemize}
            \item on ne prend pas le meilleur représentant
            \item trop restrictif ?
        \end{itemize}
    \end{frame}

    \begin{frame}{méthodes prédéfinies}
        On défini à l'avance les classes, on classe en fonction des critères prédéfinis.\\
        \emph{Avantages:}
        \begin{itemize}
            \item On paramétrize le niveau de contrainte et précision des classes
        \end{itemize}
        \emph{Inconvénients:}
        \begin{itemize}
            \item On peut masquer / oublier une classe
        \end{itemize}
    \end{frame}

    \begin{frame}{méthode du meilleur représentant}
        Méthode du représentant mais on itère récursivement dans la classe pour prendre le meilleur représentant.

        \emph{Avantage:}
        \begin{itemize}
            \item Prend a priori le meilleur représentant
            \item produit des classses plus générales ?
        \end{itemize}
        \emph{Inconvénients:}
        \begin{itemize}
            \item Trop grosses classes ?
        \end{itemize}
    \end{frame}

    \section{Betweenness vs fréquence}

    \begin{frame}{Corrélation des deux mesures}
    \begin{itemize}
        \item $\epsilon = 0.01$: Correlation: $0.1$
        \item $\epsilon = 0.03$: Correlation: $0.06$
        \item $\epsilon = 0.1$:  Correlation: $0.03$
    \end{itemize}
    \end{frame}

% 0 443
% 1 22
% 2 130
% 3 10
% 4 193
% 5 8
% 6 52
% 7 36
% 8 44
% 9 3
% 10 14
% 11 2
% 12 3
% 13 2
% 14 19
% 15 3
% 16 1
% 17 2
% 18 1
% 19 1
% 20 2
% 21 1
% 22 2
% 23 1
% 24 1
% 25 1
% 26 1
% 27 1
% 28 1
\end{document}

