\documentclass[aspectratio=169]{beamer}
\usetheme{metropolis}

\usepackage{roboto}
\usepackage{mathtools}
\usepackage{fixmath}
\usepackage{graphicx}
\usepackage{tikz}
\usepackage{stmaryrd}
\usepackage{svg}
\usepackage{pdfpages}
\usepackage{algpseudocode}
\usepackage{algorithm}
\usepackage{multicol}

\graphicspath{ {../images/lanes/} }
\setbeamertemplate{navigation symbols}{}
\setbeamercolor{background canvas}{bg=}

\newcommand{\N}{\mathbb{N}}
\DeclarePairedDelimiter\Brackets{\llbracket}{\rrbracket}

\title{Analyses de Robustesse: Graphe valué par le nombre de voies et zone sans pont}
\subtitle{Casser des Graphes}
\author{Louis Milhaud}
\institute{Complex Networks - LIP6}
\date{\today}

\begin{document}

    \AtBeginSection[]{
    \begin{frame}
    \vfill
    \centering
    \begin{beamercolorbox}[sep=8pt,center,shadow=true,rounded=true]{title}
        \usebeamerfont{title}\insertsectionhead\par%
    \end{beamercolorbox}
    \vfill
    \end{frame}
    }
    % frame 1
    \begin{frame}
        \titlepage
    \end{frame}

    % frame 2
    \begin{frame}
        \frametitle{Outline}
        \tableofcontents
    \end{frame}

    \section{Quelques scores et distributions}

    \begin{frame}
        \frametitle{Score: moyenne egde Betweenness Centrality}
        \centering
        \includegraphics[scale=0.6]{others/avg_bc_scores_lanes.pdf}
    \end{frame}

    \begin{frame}
        \frametitle{Score: plus grande composante connexe}
        \centering
        \includegraphics[scale=0.6]{others/cc_scores_lanes.pdf}
    \end{frame}

    \begin{frame}
        \frametitle{Score: plus grande composante fortement connexe}
        \centering
        \includegraphics[scale=0.6]{others/biggest_scc_lanes.pdf}
    \end{frame}

    \begin{frame}
        \frametitle{Ordre eBC: changements de distributions après 25 arêtes enlevées}
        \centering
        \includegraphics[scale=0.65]{others/lanes_bcdistr_0-25_bc.pdf}    
    \end{frame}

    \begin{frame}
        \frametitle{Ordre deg: changements de distributions après 10 arêtes enlevées}
        \centering
        \includegraphics[scale=0.65]{others/lanes_bcdistr_0-10_deg.pdf}    
    \end{frame}

    \begin{frame}
        \frametitle{Ordre freq: changements de distributions après 12 arêtes enlevées}
        \centering
        \includegraphics[scale=0.65]{others/lanes_bcdistr_0-12_freq.pdf}    
    \end{frame}

    \begin{frame}
        \frametitle{Ordre freq: changements de distributions après 25 arêtes enlevées}
        \centering
        \includegraphics[scale=0.65]{others/lanes_bcdistr_0-25_freq.pdf}    
    \end{frame}

    \begin{frame}
        \frametitle{Visuel eBC de base}
        \centering
        \includegraphics[trim=0 0 0 7cm, scale=0.7]{others/visubc_0.pdf}
    \end{frame}

    \section{Ordre eBC}

    \section{Ordre freq}

    \section{Ordre eBC sur une coupe particulière}

    \section{zones sans pont}
    \begin{frame}
        \frametitle{Quelques essais de zones: 100m (1)}
        \includegraphics{../other/test100.pdf}
    \end{frame}

    \begin{frame}
        \frametitle{Quelques essais de zones: 200m (2)}
        \includegraphics{../other/test200.pdf}
    \end{frame}

    \begin{frame}
        \frametitle{Quelques essais de zones: 300m (3)}
        \includegraphics{../other/test300.pdf}
    \end{frame}

    \end{document}