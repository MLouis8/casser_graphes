\documentclass[aspectratio=169]{beamer}
\usetheme{metropolis}

\usepackage{roboto}
\usepackage{mathtools}
\usepackage{fixmath}
\usepackage{graphicx}
\usepackage{tikz}
\usepackage{stmaryrd}
\usepackage{svg}
\usepackage{pdfpages}
\usepackage{algpseudocode}
\usepackage{algorithm}
\usepackage{multicol}

\graphicspath{ {../images/} }
\setbeamertemplate{navigation symbols}{}
\setbeamercolor{background canvas}{bg=}

\newcommand{\N}{\mathbb{N}}
\DeclarePairedDelimiter\Brackets{\llbracket}{\rrbracket}

\title{Résultats de classification, comparaison des différentes fonctions de coût}
\subtitle{Stage Casser des Graphes}
\author{Louis Milhaud}
\institute{Complex Networks - LIP6}
\date{\today}

\begin{document}

    \AtBeginSection[]{
    \begin{frame}
    \vfill
    \centering
    \begin{beamercolorbox}[sep=8pt,center,shadow=true,rounded=true]{title}
        \usebeamerfont{title}\insertsectionhead\par%
    \end{beamercolorbox}
    \vfill
    \end{frame}
    }
    % frame 1
    \begin{frame}
        \titlepage
    \end{frame}

    % frame 2
    \begin{frame}
        \frametitle{Outline}
        \tableofcontents
    \end{frame}

    \section{Résultats de classification}
    \begin{frame}
        \frametitle{Clustering avec intersection, level 0 (0.003)}
        \centering
        \includegraphics[scale=0.75]{clusters/clusters003_inter.pdf}
    
    \end{frame}
    
    \begin{frame}
        \frametitle{Clustering avec intersection, level 1 (0.003)}
        \centering
        \includegraphics[scale=0.7]{clusters/clusters003_interbis.pdf}
    \end{frame}
    
    \begin{frame}
        \frametitle{Clustering avec distance geographique: conclusions}
        \begin{itemize}
            \item Quelques soit les méthodes (max, sum, var, mean, sum des carrés) on a un résultat équivalent
            \item Par contre ce n'est intéressant qu'avec un treshold sinon il se passe ça:
        \end{itemize}
        \centering
        \includegraphics[scale=0.5]{clusters/clusters003_sum.pdf}
    \end{frame}
    
    \begin{frame}
        \frametitle{Clustering avec distance geographique: résultats}
        \centering
        \includegraphics[scale=0.75]{clusters/cluster_t_50000.pdf}
    \end{frame}

    \subsection{Outils gardés}
    \subsection{Quelques résultats}
    \section{fonctions de coûts}
    
\end{document}