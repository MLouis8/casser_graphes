\documentclass[aspectratio=169]{beamer}
\usetheme{metropolis}

\usepackage{roboto}
\usepackage{mathtools}
\usepackage{fixmath}
\usepackage{graphicx}
\usepackage{tikz}
\usepackage{stmaryrd}
\usepackage{svg}
\usepackage{pdfpages}
\usepackage{algpseudocode}
\usepackage{algorithm}
\usepackage{multicol}

\graphicspath{ {../images/} }
\setbeamertemplate{navigation symbols}{}
\setbeamercolor{background canvas}{bg=}

\newcommand{\N}{\mathbb{N}}
\DeclarePairedDelimiter\Brackets{\llbracket}{\rrbracket}

\title{Résultats de classification, comparaison des différentes fonctions de coût}
\subtitle{Stage Casser des Graphes}
\author{Louis Milhaud}
\institute{Complex Networks - LIP6}
\date{\today}

\begin{document}

    \AtBeginSection[]{
    \begin{frame}
    \vfill
    \centering
    \begin{beamercolorbox}[sep=8pt,center,shadow=true,rounded=true]{title}
        \usebeamerfont{title}\insertsectionhead\par%
    \end{beamercolorbox}
    \vfill
    \end{frame}
    }
    % frame 1
    \begin{frame}
        \titlepage
    \end{frame}

    % frame 2
    \begin{frame}
        \frametitle{Outline}
        \tableofcontents
    \end{frame}

    \section{Résultats de classification}
    
    \begin{frame}
        \frametitle{Clustering avec intersection, level 1 (0.003)}
        \centering
        \includegraphics[scale=0.7]{clusters/clusters003_interbis.pdf}
    \end{frame}
    
    \begin{frame}
        \frametitle{Clustering avec distance geographique: conclusions}
        \begin{itemize}
            \item Quelques soit les méthodes (max, sum, var, mean, sum des carrés) on a un résultat équivalent
            \item Par contre ce n'est intéressant qu'avec un treshold sinon il se passe ça:
        \end{itemize}
        \centering
        \includegraphics[scale=0.5]{clusters/clusters003_sum.pdf}
    \end{frame}
    
    \begin{frame}
        \frametitle{Clustering avec distance geographique: résultats}
        \centering
        \includegraphics[scale=0.75]{clusters/cluster_t_50000.pdf}
    \end{frame}

    \section{Coût}
    \begin{frame}
        \frametitle{Retour sur la width}
        Anciennement:
        \begin{itemize}
            \item inférée à partir de lanes (97\%)
            \item lanes inféré à partir de highway (49\%)
            \begin{itemize}
                \item on regarde la distribution des lanes par highway (51\%)
                \item on prend le max de la catégorie
            \end{itemize}
            On suppose donc que la répartition des lanes est aléatoire et on ne la suit pas avec précision.
        \end{itemize}    
    \end{frame}

    \begin{frame}
        \frametitle{Répartition des lanes}
        \centering
        \includegraphics[scale=0.4]{cost_def/edges_with_lanes.pdf}
        \includegraphics[scale=0.4]{cost_def/edges_without_lanes.pdf}
    \end{frame}
    
    \begin{frame}
        \frametitle{Pertinence de l'inférence}
        \centering
        \includegraphics[scale=0.35]{cost_def/type_distr.png}
    \end{frame}

    \begin{frame}
        \frametitle{Le critère oneway}
        \centering
        \includegraphics[scale=0.4]{cost_def/oneway.pdf}
        \includegraphics[scale=0.4]{cost_def/not_oneway.pdf}
    
    \end{frame}
    \begin{frame}
        \frametitle{Affinage de l'inférence}
        Le calcul de la width devient:
        \begin{itemize}
            \item 3\% width
            \item inféré à partir de lane si il y a ($\approx$50\%)
            \item le reste des lanes inféré par rapport à la distribution des highway et des oneway
        \end{itemize}
        On enlève pas le biais de la répartition des lanes renseigné mais on suit plus finement la distribution.
    \end{frame}

    \section{Analyses de nouvelles coupes}
    \begin{frame}
        \frametitle{Comparaison ancienne et nouvelle width}
    
        % heatmaps et clusters de coupes
    
    \end{frame}

\end{document}