\documentclass[aspectratio=169]{beamer}
\usetheme{metropolis}

\usepackage{roboto}
\usepackage{mathtools}
\usepackage{fixmath}
\usepackage{graphicx}
\usepackage{tikz}
\usepackage{stmaryrd}
\usepackage{svg}

\graphicspath{ {../../images/} }
\setbeamertemplate{navigation symbols}{}

\newcommand{\N}{\mathbb{N}}
\DeclarePairedDelimiter\Brackets{\llbracket}{\rrbracket}

\title{Analyse de coupe: coupe non valuée}
\subtitle{Stage Casser des Graphes}
\author{Louis Milhaud}
\institute{Complex Networks - LIP6}
\date{\today}

\begin{document}

    \begin{frame}
        \titlepage
    \end{frame}

    \begin{frame}
        \frametitle{Statistiques de base}
    
        \begin{itemize}
            \item La meilleure coupe coupe 39 arêtes
            \item Elle apparaît avec une fréquence de 0.001
            \item Pour 1000 coupes on a une moyenne de 45
            \item Et une std de 2.3
        \end{itemize}
    
    \end{frame}

    \begin{frame}
        \frametitle{Classes}
        \centering
        \includegraphics[scale=0.75]{clusters/CTS_nocost_005.pdf}    
    \end{frame}

    \begin{frame}
        \frametitle{Statistiques sur les fréquences}
        
        \begin{itemize}
            \item L'arête la plus coupée (7550, 7552) l'a été 530 fois
            \item La moins 1 (parmis celles coupées)
            \item Pour une moyenne de coupe de 12
            \item Et une std de 29
            \item Pour 1000 coupes on coupe 3903 arêtes différentes (sur 46761)
        \end{itemize}
        
    \end{frame}

    \begin{frame}
        \frametitle{10 arêtes les plus coupées}
        \centering
        \includegraphics[scale=1.1]{nocost_10_most_cut.pdf}    
    \end{frame}

\end{document}