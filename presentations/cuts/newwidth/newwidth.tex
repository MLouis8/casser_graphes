\documentclass[aspectratio=169]{beamer}
\usetheme{metropolis}

\usepackage{roboto}
\usepackage{mathtools}
\usepackage{fixmath}
\usepackage{graphicx}
\usepackage{tikz}
\usepackage{stmaryrd}
\usepackage{svg}

\graphicspath{ {../../images/} }
\setbeamertemplate{navigation symbols}{}

\newcommand{\N}{\mathbb{N}}
\DeclarePairedDelimiter\Brackets{\llbracket}{\rrbracket}

\title{Analyse de coupe: largeur}
\subtitle{Stage Casser des Graphes}
\author{Louis Milhaud}
\institute{Complex Networks - LIP6}
\date{\today}

\begin{document}

    \begin{frame}
        \titlepage
    \end{frame}

    \section{Width}
    \begin{frame}
        \frametitle{Statistiques de base}

        \begin{itemize}
            \item La meilleure coupe a un coût de 320
            \item Elle apparaît avec une fréquence de 0.001
            \item La pire a un coût de 400
            \item Pour 1000 coupes on a une moyenne de 349
            \item Et une std de 15
        \end{itemize}
    
    \end{frame}

    \begin{frame}
        \frametitle{Classes}
        \centering
        \includegraphics[scale=0.7]{clusters/CTS_newwidth.pdf}
    \end{frame}

    \begin{frame}
        \frametitle{Statistiques sur les fréquences}
        
        \begin{itemize}
            \item L'arête la plus coupée (27074, 34591) l'a été 764 fois
            \item La moins 1 (parmis celles coupées)
            \item Pour une moyenne de coupe de 15
            \item Et une std de 43
            \item Pour 1000 coupes on coupe 3572 arêtes différentes (sur 46761)
        \end{itemize}
    
    \end{frame}

    \begin{frame}
        \frametitle{10 arêtes les plus coupées}
        \centering
        \includegraphics[scale=1.1]{width_10_most_cut.pdf}
    \end{frame}

    \section{Width squarred}

    \begin{frame}
        \frametitle{Statistiques de base}

        \begin{itemize}
            \item La meilleure coupe a un coût de 2557
            \item Elle apparaît avec une fréquence de 0.003
            \item La pire a un coût de 3296
            \item Pour 1000 coupes on a une moyenne de 2744
            \item Et une std de 84
        \end{itemize}
    
    \end{frame}

    \begin{frame}
        \frametitle{Classes}
    
        
    
    \end{frame}

    \begin{frame}
        \frametitle{Statistiques sur les fréquences}
        
        \begin{itemize}
            \item L'arête la plus coupée (27074, 34591) l'a été 648 fois
            \item La moins 1 (parmis celles coupées)
            \item Pour une moyenne de coupe de 21
            \item Et une std de 45
            \item Pour 1000 coupes on coupe 3689 arêtes différentes (sur 46761)
        \end{itemize}
    
    \end{frame}

    \section{Width Maxspeed}
    \begin{frame}
        \frametitle{Statistiques de base}
        
        \begin{itemize}
            \item La meilleure coupe a un coût de 701
            \item Elle apparaît avec une fréquence de 0.001
            \item La pire a un coût de 787
            \item Pour 1000 coupes on a une moyenne de 739
            \item Et une std de 14
        \end{itemize}
    
    \end{frame}

    \begin{frame}
        \frametitle{Classes}
    
        
    
    \end{frame}

    \begin{frame}
        \frametitle{Statistiques sur les fréquences}
        
        \begin{itemize}
            \item L'arête la plus coupée (1273, 1274) l'a été 548 fois
            \item La moins 1 (parmis celles coupées)
            \item Pour une moyenne de coupe de 16
            \item Et une std de 36
            \item Pour 1000 coupes on coupe 8199 arêtes différentes (sur 46761)
        \end{itemize}
    
    \end{frame}

    \section{Width no tunnel}
    \begin{frame}
        \frametitle{Statistiques de base}
        
        \begin{itemize}
            \item La meilleure coupe a un coût de 324
            \item Elle apparaît avec une fréquence de 0.003
            \item La pire a un coût de 438
            \item Pour 1000 coupes on a une moyenne de 362
            \item Et une std de 18
        \end{itemize}
    
    \end{frame}

    \begin{frame}
        \frametitle{Classes}
        \centering
        \includegraphics[scale=0.75]{clusters/CTS_widthnotunnel.pdf}    
    \end{frame}

    \begin{frame}
        \frametitle{Statistiques sur les fréquences}

        \begin{itemize}
            \item L'arête la plus coupée (27074, 34591) l'a été 547 fois
            \item La moins 1 (parmis celles coupées)
            \item Pour une moyenne de coupe de 12
            \item Et une std de 28
            \item Pour 1000 coupes on coupe 4554 arêtes différentes (sur 46761)
        \end{itemize}
    
    \end{frame}

    \begin{frame}
        \frametitle{10 arêtes les plus coupées}
        \centering
        \includegraphics[scale=1.1]{widthnotunnel_10_most_cut.pdf}    
    \end{frame}
    \section{Width no bridge}
    \begin{frame}
        \frametitle{Statistiques de base}
        
        \begin{itemize}
            \item La meilleure coupe a un coût de 348
            \item Elle apparaît avec une fréquence de 0.007
            \item La pire a un coût de 436
            \item Pour 1000 coupes on a une moyenne de 384
            \item Et une std de 15
        \end{itemize}

    \end{frame}

    \begin{frame}
        \frametitle{Classes}
        \centering
        \includegraphics[scale=0.75]{clusters/CTS_widthnobridge.pdf}
    \end{frame}

    \begin{frame}
        \frametitle{Statistiques sur les fréquences}
        
        \begin{itemize}
            \item L'arête la plus coupée (2450, 25206) l'a été 314 fois
            \item La moins 1 (parmis celles coupées)
            \item Pour une moyenne de coupe de 13
            \item Et une std de 27
            \item Pour 1000 coupes on coupe 4438 arêtes différentes (sur 46761)
        \end{itemize}
    
    \end{frame}

    \begin{frame}
        \frametitle{10 arêtes les plus coupées}
        \centering
        \includegraphics[scale=1.1]{widthnobridge_10_most_cut.pdf}
    \end{frame}
    \section{Random between}

    \section{Random Following original width distribution}
\end{document}