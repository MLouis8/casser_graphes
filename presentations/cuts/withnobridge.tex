\documentclass[aspectratio=169]{beamer}
\usetheme{metropolis}

\usepackage{roboto}
\usepackage{mathtools}
\usepackage{fixmath}
\usepackage{graphicx}
\usepackage{tikz}
\usepackage{stmaryrd}
\usepackage{svg}

\graphicspath{ {../images/} }
\setbeamertemplate{navigation symbols}{}

\newcommand{\N}{\mathbb{N}}
\DeclarePairedDelimiter\Brackets{\llbracket}{\rrbracket}

\title{Analyse de coupe: largeur sans pont}
\subtitle{Stage Casser des Graphes}
\author{Louis Milhaud}
\institute{Complex Networks - LIP6}
\date{\today}

\begin{document}

    \begin{frame}
        \titlepage
    \end{frame}

    \begin{frame}
        \frametitle{Statistiques de base}
        
        \begin{itemize}
            \item La meilleure coupe a un coût de 348
            \item Elle apparaît avec une fréquence de 0.007
            \item La pire a un coût de 436
            \item Pour 1000 coupes on a une moyenne de 384
            \item Et une std de 15
        \end{itemize}

    \end{frame}

    \begin{frame}
        \frametitle{Classes}
        \centering
        \includegraphics[scale=0.75]{clusters/CTS_widthnobridge.pdf}
    \end{frame}

    \begin{frame}
        \frametitle{Statistiques sur les fréquences}
        
        \begin{itemize}
            \item L'arête la plus coupée (2450, 25206) l'a été 314 fois
            \item La moins 1 (parmis celles coupées)
            \item Pour une moyenne de coupe de 13
            \item Et une std de 27
            \item Pour 1000 coupes on coupe 4438 arêtes différentes (sur 46761)
        \end{itemize}
    
    \end{frame}

    \begin{frame}
        \frametitle{10 arêtes les plus coupées}
        \centering
        \includegraphics[scale=1.1]{widthnobridge_10_most_cut.pdf}
    \end{frame}

\end{document}