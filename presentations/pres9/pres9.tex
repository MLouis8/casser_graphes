\documentclass[aspectratio=169]{beamer}
\usetheme{metropolis}

\usepackage{roboto}
\usepackage{mathtools}
\usepackage{fixmath}
\usepackage{graphicx}
\usepackage{tikz}
\usepackage{stmaryrd}
\usepackage{svg}
\usepackage{pdfpages}
\usepackage{algpseudocode}
\usepackage{algorithm2e}
\usepackage{multicol}

\graphicspath{ {../images/} }
\setbeamertemplate{navigation symbols}{}
\setbeamercolor{background canvas}{bg=}

\newcommand{\N}{\mathbb{N}}
\DeclarePairedDelimiter\Brackets{\llbracket}{\rrbracket}

\title{Autres villes, Communautés, Approx, Effective resistance et quelques clustering}
\subtitle{Casser des Graphes}
\author{Louis Milhaud}
\institute{Complex Networks - LIP6}
\date{\today}

\begin{document}

    \AtBeginSection[]{
    \begin{frame}
    \vfill
    \centering
    \begin{beamercolorbox}[sep=8pt,center,shadow=true,rounded=true]{title}
        \usebeamerfont{title}\insertsectionhead\par%
    \end{beamercolorbox}
    \vfill
    \end{frame}
    }
    \begin{frame}
        \titlepage
    \end{frame}

    \begin{frame}
        \frametitle{Outline}
        \tableofcontents
    \end{frame}

    \section{Autres Villes}
    \begin{frame}
        \frametitle{Shanghai}
    
        
    
    \end{frame}

    \begin{frame}
        \frametitle{Manhattan}
    
        
    
    \end{frame}
    \section{Communities}
    \begin{frame}
        \frametitle{Communautés: un autre moyen de trouver des arêtes importantes}
        \centering
        \includegraphics[scale=0.8]{other/pres9/coupe_louvain.pdf}
        cost: ?
    \end{frame}

    \section{Effective Resistance}
    \begin{frame}
        \frametitle{Definition}
            The \textbf{effective resistance} between two vertices is the electrical resistance measured with the Kirchoff's circuit laws. The \textbf{effective graph resistance} is the sum of all effective resistance (for all pairs of vertices).
            \vspace{10pt}\\
            \emph{Kirchoff laws:}
            \begin{itemize}
                \item[-] $current\ in = current\ out$
                \item[-] $voltages\ of\ loop = 0 $
            \end{itemize}
            (computed from Laplacian eigenvalues: O($n^3$))
    \end{frame}
    
    \begin{frame}
        \frametitle{Premiers résultats de effective resistance}
    
    \end{frame}
    
    \section{Grandes attaques}
    \subsection{Choix}
    \begin{frame}
        \frametitle{Choix de l'approximation eBC}
        \includegraphics[scale=0.33]{approx/visubcapprox_500.pdf}
        \includegraphics[scale=0.33]{approx/visubcapprox_1000.pdf}
    \end{frame}

    \begin{frame}
        \frametitle{Choix du nombre de coupes}
        \textbf{Temps sur ma machine pour $n$ coupes:}\\
        \vspace{5pt}
        \begin{tabular}{|c|l|c|r|} 
            \hline
                $n$ & 1 & 200 & 1000 \\
            \hline
                2 blocs & & &\\
            \hline
                3 blocs & & &\\
            \hline
                imb = 0.05 & & &\\
            \hline
                imb = 0.1 & & &\\
            \hline
                imb = 0.2 & & &\\
            \hline
                imb = 0.3 & & &\\
            \hline
                temps (sec) & D1 & D2 & D3 \\ 
            \hline
            \end{tabular}
    \end{frame}

    \subsection{Nouvelles coupes}
    \begin{frame}
        \frametitle{déséquilibre grand: imb = 0.1}
    
    \end{frame}
    
    \begin{frame}
        \frametitle{déséquilibre grand: imb = 0.3}
    
    \end{frame}
    
    \begin{frame}
        \frametitle{déséquilibre grand: imb = 0.3}
    
    \end{frame}

    \begin{frame}
        \frametitle{nblocks = 3}
    
    \end{frame}
    
    \end{document}