\documentclass[aspectratio=169]{beamer}
\usetheme{metropolis}

\usepackage{roboto}
\usepackage{mathtools}
\usepackage{fixmath}
\usepackage{graphicx}
\usepackage{tikz}
\usepackage{stmaryrd}
\usepackage{svg}
\usepackage{pdfpages}
\usepackage{algpseudocode}
\usepackage{algorithm2e}
\usepackage{multicol}

\graphicspath{ {../images/} }
\setbeamertemplate{navigation symbols}{}
\setbeamercolor{background canvas}{bg=}

\newcommand{\N}{\mathbb{N}}
\DeclarePairedDelimiter\Brackets{\llbracket}{\rrbracket}

\title{Autres villes, Communautés, Approx, Effective resistance, Nouveaux clusters, Analyse de robustesse}
\subtitle{Casser des Graphes}
\author{Louis Milhaud}
\institute{Complex Networks - LIP6}
\date{\today}

\begin{document}

    \AtBeginSection[]{
    \begin{frame}
    \vfill
    \centering
    \begin{beamercolorbox}[sep=8pt,center,shadow=true,rounded=true]{title}
        \usebeamerfont{title}\insertsectionhead\par%
    \end{beamercolorbox}
    \vfill
    \end{frame}
    }
    \begin{frame}
        \titlepage
    \end{frame}

    \begin{frame}
        \frametitle{Outline}
        \tableofcontents
    \end{frame}

    \section{Autres Villes}
    \begin{frame}
        \frametitle{Autres villes}
        \begin{itemize}
            \item Shanghai
            \item Manhattan
        \end{itemize}
        Tolérance ?\\
        Buffer ?    
    \end{frame}

    \section{Communities}
    \begin{frame}
        \frametitle{Communautés: un autre moyen de trouver des arêtes importantes}
        \centering
        \includegraphics[scale=0.8]{other/pres9/coupe_louvain.pdf}\\
        cost: 86 edges, 196 lanes
    \end{frame}

    \section{Effective Resistance}
    \begin{frame}
        \frametitle{Definition}
            The \textbf{effective resistance} between two vertices is the electrical resistance measured with the Kirchoff's circuit laws. The \textbf{effective graph resistance} is the sum of all effective resistance (for all pairs of vertices).
            \vspace{10pt}\\
            \emph{Kirchoff laws:}
            \begin{itemize}
                \item[-] $current\ in = current\ out$
                \item[-] $voltages\ of\ loop = 0 $
            \end{itemize}
            (computed from Laplacian eigenvalues: O($n^3$))
    \end{frame}
    
    \begin{frame}
        \frametitle{Premiers résultats de effective resistance}
        \centering
        \includegraphics[scale=0.65]{other/pres9/dir1.pdf}
    \end{frame}

    \section{Analyse de robustesse sur graphe dirigé valué}
    
    \begin{frame}
        \frametitle{Strongly Connected Components}
        \centering
        \includegraphics[scale=0.42]{other/pres9/sccs2.pdf}
        \includegraphics[scale=0.42]{other/pres9/sccs1.pdf}
    \end{frame}

    \begin{frame}
        \frametitle{Ordre à l'intérieur d'une coupe}
        \centering 
        \includegraphics[scale=0.65]{other/pres9/avgeBC_efficiencyRDvsBC.pdf}
    \end{frame}
    
    
    \section{Grandes attaques}
    \begin{frame}
        \frametitle{Approx frequence}
        \centering
        \includegraphics[scale=0.65]{other/pres9/bigattacks_freq_approx_03.pdf}
    \end{frame}
    \begin{frame}
        \frametitle{Strongly Connected Components}
        \centering
        \includegraphics[scale=0.42]{other/pres9/bigattacks_scc.pdf}
        \includegraphics[scale=0.42]{other/pres9/bigattacks_scc2.pdf}
    \end{frame}
    \section{Choix}
    \begin{frame}
        \frametitle{Choix de l'approximation eBC}
        \includegraphics[scale=0.33]{approx/visubcapprox_500.pdf}
        \includegraphics[scale=0.33]{approx/visubcapprox_1000.pdf}
    \end{frame}

    \begin{frame}
        \frametitle{Choix du nombre de coupes}
        \textbf{Temps sur ma machine pour $n$ coupes:}\\
        \vspace{5pt}
        \begin{tabular}{|c|l|c|r|} 
            \hline
                $n$ & 1 & 200 & 1000 \\
            \hline
                imb = 0.05, nblocs = 2 &  0.5" & 1'08"& 7'30"\\
            \hline
                imb = 0.05, nblocs = 3 & 0.8" & 2'15" & 14'10"\\
            \hline
                imb = 0.1, nblocs = 2 & 0.8" & 2'30" & 13'15"\\
            \hline
                imb = 0.2, nblocs = 2 & 0.4" & 1'23" & 6'25"\\
            \hline
                imb = 0.3, nblocs = 2 & 0.45" & 1'30" & 7'04"\\
            \hline
            \end{tabular}
    \end{frame}

    \section{Clustering}
    \subsection{Clustering pour les nouvelles coupes}
    \begin{frame}
        \frametitle{Clustering: idées et questions}
        \emph{Distance:}
        Utilisation de la distance de Chamfer.\\
        Pour deux ensembles de coupes $C_1$ et $C_2$:
        $$\sum_{c_1\in C_1}\min_{c_2\in C_2}\{c_1, c_2\} + \sum_{c_2\in C_2}\min_{c_1\in C_1}\{c_1, c_2\}$$
        avec d la distance géographique.\\
        Possibilités de l'approximer, mais l'article est pas facile.
    \end{frame}

    \begin{frame}
        \frametitle{Clustering: idées et questions}
        Ensuite quel algorithme ?\\
        \begin{itemize}
            \item BIRCH $\rightarrow$ centroide ?
            \item BallTree $\rightarrow$ linéaire ?
        \end{itemize}
    \end{frame}

    \end{document}