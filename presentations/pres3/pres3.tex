\documentclass[aspectratio=169]{beamer}
\usetheme{metropolis}

\usepackage{roboto}
\usepackage{mathtools}
\usepackage{fixmath}
\usepackage{graphicx}
\usepackage{tikz}
\usepackage{stmaryrd}
\usepackage{svg}
\usepackage{pdfpages}
\usepackage{algpseudocode}
\usepackage{algorithm}
\usepackage{multicol}

\graphicspath{ {../images/} }
\setbeamertemplate{navigation symbols}{}
\setbeamercolor{background canvas}{bg=}

\newcommand{\N}{\mathbb{N}}
\DeclarePairedDelimiter\Brackets{\llbracket}{\rrbracket}

\title{L'outil de classification, relation betweenness et coupe, composantes connexes, étude des coûts}
\subtitle{Stage Casser des Graphes}
\author{Louis Milhaud}
\institute{Complex Networks - LIP6}
\date{\today}

\begin{document}

    \AtBeginSection[]{
    \begin{frame}
    \vfill
    \centering
    \begin{beamercolorbox}[sep=8pt,center,shadow=true,rounded=true]{title}
        \usebeamerfont{title}\insertsectionhead\par%
    \end{beamercolorbox}
    \vfill
    \end{frame}
    }
    % frame 1
    \begin{frame}
        \titlepage
    \end{frame}

    % frame 2
    \begin{frame}
        \frametitle{Outline}
        \tableofcontents
    \end{frame}
    \section{L'outil de classification}
    \begin{frame}
        \frametitle{Description}
        \emph{On veut pouvoir distinguer les coupes vraiment différentes visuellement}\\
        Plusieurs niveaux d'approche:
        \begin{itemize}
            \item les critères (notion de distance entre deux coupes)
            \begin{itemize}
                \item[-] intersection
                \item[-] distance dans le graphe
                \item[-] distance géographique
                \item[-] distance géométrique    
            \end{itemize}
            \item la méthode (comment ensuite trier en fonction des critères)
            \begin{itemize}
                \item[-] méthodes maisons (représentant, division)
                \item[-] clustering du graphe des distances
            \end{itemize}
        \end{itemize}
    \end{frame}

    \begin{frame}
        \frametitle{Résultats}
    
        
    
    \end{frame}
    \section{Relation Betweenness et Coupe}
    \begin{frame}
        \frametitle{Betweenness distribution entire graph vs 100 most cut edges (1)}
    
        \includegraphics[scale=0.65]{betweenness_distr.pdf}
    
    \end{frame}

    \begin{frame}
        \frametitle{Betweenness distribution entire graph vs 100 most cut edges (2)}
    
        \includegraphics[scale=0.65]{betweenness_distrbis.pdf}
    
    \end{frame}

    \section{Composantes Connexes}
    \begin{frame}
        \frametitle{Les étranges composantes connexes}
        \begin{itemize}
            \item quasiment toutes de taille 28
            \item si on les compare en enlevant les deux plus grandes composantes:
        \end{itemize}
        \includegraphics[width=0.9\paperwidth]{compcon.png}
    \end{frame}

    \section{étude des coûts}
    \begin{frame}
        \frametitle{Différents attributs et leur présence}
        \begin{multicols}{2}
            \begin{itemize}
                \item \textcolor{green}{$(n_1, n_2)$} $\qquad 100\%$
                \item \textcolor{green}{width} $\qquad\quad 3\%$ 
                \item \textcolor{green}{maxspeed} $\ 95\%$
                \item \textcolor{green}{oneway} $\quad\ 100\%$
                \item \textcolor{green}{lanes} $\qquad\ 51\%$
                \item \textcolor{green}{bridge} $\qquad\ 2\%$
                \item \textcolor{green}{tunnel} $\qquad\ 1\%$
                \item \textcolor{yellow}{highway} $\quad 100\%$
                \item \textcolor{yellow}{access}  $\qquad\ 7\%$
            \end{itemize}
            \vfill\null
            \columnbreak
            \begin{itemize}
                \item \textcolor{yellow}{reversed} $\qquad 1\%$
                \item \textcolor{yellow}{ref} $\qquad\qquad\ 6\%$
                \item \textcolor{yellow}{junction} $\qquad 0.5\%$
                \item \textcolor{yellow}{service} $\qquad\ 0.01\%$
                \item \textcolor{red}{length}
                \item \textcolor{red}{name}
                \item \textcolor{red}{v\_original}
                \item \textcolor{red}{u\_original}
                \item \textcolor{red}{osmid}
            \end{itemize}
            \vfill\null
        \end{multicols}
    \end{frame}
    \begin{frame}
        \frametitle{Quelques idées de coûts}
        \begin{itemize}
            \item width
            \item width$^2$
            \item width avec maxspeed 50
            \item width sans bridge
            \item width sans tunnel
            \item random(min, max)
            \item random distribution
        \end{itemize}
    \end{frame}
\end{document}