\documentclass[aspectratio=169]{beamer}
\usetheme{metropolis}

\usepackage{roboto}
\usepackage{mathtools}
\usepackage{fixmath}
\usepackage{graphicx}
\usepackage{tikz}
\usepackage{stmaryrd}
\usepackage{svg}
\usepackage{pdfpages}
\usepackage{algpseudocode}
\usepackage{algorithm}
\usepackage{multicol}

\graphicspath{ {../images/} }
\setbeamertemplate{navigation symbols}{}
\setbeamercolor{background canvas}{bg=}

\newcommand{\N}{\mathbb{N}}
\DeclarePairedDelimiter\Brackets{\llbracket}{\rrbracket}

\title{Analyses de Robustesse: procédure pour analyser et comparer les stratégies}
\subtitle{Casser des Graphes}
\author{Louis Milhaud}
\institute{Complex Networks - LIP6}
\date{\today}

\begin{document}

    \AtBeginSection[]{
    \begin{frame}
    \vfill
    \centering
    \begin{beamercolorbox}[sep=8pt,center,shadow=true,rounded=true]{title}
        \usebeamerfont{title}\insertsectionhead\par%
    \end{beamercolorbox}
    \vfill
    \end{frame}
    }
    \begin{frame}
        \titlepage
    \end{frame}

    \begin{frame}
        \frametitle{Outline}
        \tableofcontents
    \end{frame}

    \section{Stratégies}
    \subsection{Stratégie edge Betweenness Centrality (eBC)}
    \begin{frame}
        \frametitle{Stratégie edge Betweenness Centrality (eBC): Définition}
        \begin{algorithmic}
            de
        \end{algorithmic}
    \end{frame}
    \subsection{Stratégie fréquence (freq)}
    \subsection{Stratégie random (rd)}
    \subsection{Stratégie coupe (cut)}
    \section{Comparaison globale}
    \subsection{Métriques de robustesse classique}
    \subsection{Dérivées de l'eBC}
    \section{Visualisation de l'attaque}
    \subsection{Attaque entière}
    \subsection{Une attaque plus réaliste}
    \end{document}