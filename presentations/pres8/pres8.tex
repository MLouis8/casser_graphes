\documentclass[aspectratio=169]{beamer}
\usetheme{metropolis}

\usepackage{roboto}
\usepackage{mathtools}
\usepackage{fixmath}
\usepackage{graphicx}
\usepackage{tikz}
\usepackage{stmaryrd}
\usepackage{svg}
\usepackage{pdfpages}
\usepackage{algpseudocode}
\usepackage{algorithm2e}
\usepackage{multicol}

\graphicspath{ {../images/} }
\setbeamertemplate{navigation symbols}{}
\setbeamercolor{background canvas}{bg=}

\newcommand{\N}{\mathbb{N}}
\DeclarePairedDelimiter\Brackets{\llbracket}{\rrbracket}

\title{Analyses de Robustesse: procédure pour analyser et comparer les stratégies}
\subtitle{Casser des Graphes}
\author{Louis Milhaud}
\institute{Complex Networks - LIP6}
\date{\today}

\begin{document}

    \AtBeginSection[]{
    \begin{frame}
    \vfill
    \centering
    \begin{beamercolorbox}[sep=8pt,center,shadow=true,rounded=true]{title}
        \usebeamerfont{title}\insertsectionhead\par%
    \end{beamercolorbox}
    \vfill
    \end{frame}
    }
    \begin{frame}
        \titlepage
    \end{frame}

    \begin{frame}
        \frametitle{Outline}
        \tableofcontents
    \end{frame}

    \section{Stratégies}
    \begin{frame}
        \frametitle{Stratégies d'attaque de base}
        \begin{itemize}
            \item \textbf{Ordre edge Betweenness Centrality (eBC):}\\
            On calcule la eBC de tout le graphe, on enlève l'arête à plus grande eBC et on recommence.
            \item \textbf{Ordre degré max (deg):}\\
            On enlève l'arête de plus fort degré ($deg(a, b) = deg(a) \times deg(b)$) et on recommence.
            \item \textbf{Ordre random (rd):}\\
            On enlève une arête au hasard et on recommence. 
        \end{itemize}
    \end{frame}

    \begin{frame}
        \frametitle{Stratégies d'attaque basées sur KaHIP}
        \begin{itemize}
            \item \textbf{Ordre fréquence (freq):}\\
            On coupe $1000$ fois le graph avec Kaffpa, on enlève l'arête la plus coupée et on recommence.
            \item \textbf{Attaque sur une coupe (cut):}\\
            On applique un ordre de base (slide précédente) sur une coupe du graphe.
            \item \textbf{Ordre fréquence amélioré (cluster):}\\
            On applique l'ordre fréquence sur un cluster de coupe.
        \end{itemize}
    \end{frame}
    \section{Comparaison globale}

    \begin{frame}
        \frametitle{Métriques de robustesse classique}
        \begin{itemize}
            \item eBC moyenne \textbf{(avg eBC)}
            \vspace{20pt}
            \item efficacité \textbf{(global efficiency)}
            \vspace{20pt}
            \item taille de la plus grande composante connexe \textbf{(scc)}
        \end{itemize}
    \end{frame}

    \begin{frame}
        \frametitle{Métriques de robustesse dérivées de l'eBC}
        \textbf{\textit{Perturbation:}} changement de valeur de eBC.
        \begin{itemize}
            \item nombre d'arêtes perturbées \textbf{(nimpacts)}
            \vspace{5pt}
            \item distance de l'arête la plus loin de l'attaque \textbf{(dist)}
            \vspace{5pt}
            \item somme des perturbations (en valeur absolue) \textbf{(sumdiffs)}
            \vspace{5pt}
            \item somme des perturbations / distance \textbf{(sumbydist)}
        \end{itemize}
        \vspace{10pt}
        Ces métriques d'impacts sont adaptable avec toute métrique valuant les arêtes.
    \end{frame}

    \subsection{Métriques de robustesse classique}
    
    \begin{frame}
        \frametitle{Comparaison de l'avg eBC et efficiency sur quelques stratégies}
        \centering
        \includegraphics[scale=0.6]{other/avgbc_efficiency_NW.pdf}
    \end{frame}

    \begin{frame}
        \frametitle{Comparaison de la scc sur quelques stratégies}
        \centering
        \includegraphics[scale=0.6]{other/scc_NW.pdf}
    \end{frame}

    \subsection{Dérivées de l'eBC}
    
    \begin{frame}
        \frametitle{Comparaison de nimpacts sur quelques stratégies}
        \centering
        \includegraphics[scale=0.6]{impacts/nimpacts_NW.pdf}
    \end{frame}

    \begin{frame}
        \frametitle{Comparaison de dmax sur quelques stratégies}
        \centering
        \includegraphics[scale=0.6]{impacts/dmax_NW.pdf}
    \end{frame}

    \begin{frame}
        \frametitle{Comparaison de sumdiffs sur quelques stratégies}
        \centering
        \includegraphics[scale=0.6]{impacts/sumdiffs_NW.pdf}
    \end{frame}

    \begin{frame}
        \frametitle{Comparaison de sumbydist sur quelques stratégies}
        \centering
        \includegraphics[scale=0.6]{impacts/sumbydist_NW.pdf}
    \end{frame}

    \begin{frame}
        \frametitle{Comparaison sumdiffs non cumulée et la distance}
        \centering
        \includegraphics[scale=0.6]{impacts/scatterplot_NW.pdf}
    \end{frame}

    \section{Analyse locale}
    \subsection{Attaque entière}
    
    \begin{frame}
        \frametitle{Avant / Après stratégie eBC}
        \includegraphics[scale=0.22]{lanes/bc/triplevisu1.pdf}
        \includegraphics[scale=0.22]{lanes/bc/triplevisu3.pdf}
        \includegraphics[scale=0.22]{lanes/bc/triplevisu2.pdf}
    \end{frame}

    \begin{frame}
        \frametitle{Avant / Après stratégie freq}
        \includegraphics[scale=0.22]{lanes/freq/triplevisu1.pdf}
        \includegraphics[scale=0.22]{lanes/freq/triplevisu3.pdf}
        \includegraphics[scale=0.22]{lanes/freq/triplevisu2.pdf}
    \end{frame}

    \begin{frame}
        \frametitle{Avant / Après stratégie deg}
        \includegraphics[scale=0.22]{lanes/deg/triplevisu1.pdf}
        \includegraphics[scale=0.22]{lanes/deg/triplevisu3.pdf}
        \includegraphics[scale=0.22]{lanes/deg/triplevisu2.pdf}
    \end{frame}

    \subsection{Une attaque plus réaliste}
    \begin{frame}
        \frametitle{3 premières arêtes: comparaison des stratégie de base}
        \includegraphics[scale=0.22]{lanes/bc/triplevisubis3.pdf}
        \includegraphics[scale=0.22]{lanes/freq/triplevisubis3.pdf}
        \includegraphics[scale=0.22]{lanes/deg/triplevisubis3.pdf}
    \end{frame}
    \end{document}