\documentclass[aspectratio=169]{beamer}
\usetheme{metropolis}

\usepackage{roboto}
\usepackage{mathtools}
\usepackage{fixmath}
\usepackage{graphicx}
\usepackage{tikz}
\usepackage{stmaryrd}
\usepackage{svg}
\usepackage{pdfpages}
\usepackage{algpseudocode}
\usepackage{algorithm}
\usepackage{multicol}

\graphicspath{ {../images/} }
\setbeamertemplate{navigation symbols}{}
\setbeamercolor{background canvas}{bg=}

\newcommand{\N}{\mathbb{N}}
\DeclarePairedDelimiter\Brackets{\llbracket}{\rrbracket}

\title{(non) Amélioration du preprocessing, Nouvelle fonction de coût, robustesse, composantes connexes en fonction du déséquilibre}
\subtitle{Casser des Graphes}
\author{Louis Milhaud}
\institute{Complex Networks - LIP6}
\date{\today}

\begin{document}

    \AtBeginSection[]{
    \begin{frame}
    \vfill
    \centering
    \begin{beamercolorbox}[sep=8pt,center,shadow=true,rounded=true]{title}
        \usebeamerfont{title}\insertsectionhead\par%
    \end{beamercolorbox}
    \vfill
    \end{frame}
    }
    % frame 1
    \begin{frame}
        \titlepage
    \end{frame}

    % frame 2
    \begin{frame}
        \frametitle{Outline}
        \tableofcontents
    \end{frame}

    \section{Amélioration du preprocessing}
    \begin{frame}
        \frametitle{fusion des edges attributes}
        \centering
        \includegraphics[scale=0.4]{fusion_attributs.png}
    \end{frame}
    \begin{frame}
        \frametitle{Ajoût des lignes de bus}
        impossible car données non uniformes:\\
        \vspace{15pt}
        \begin{center}
            \includegraphics[scale=0.25]{Capture d’écran 2024-04-08 à 10.03.23.png}
            \includegraphics[scale=0.25]{Capture d’écran 2024-04-08 à 10.05.07.png}
        \end{center}
        Première image: $lanes = 2$ et pas de busway\\
        Seconde image: $lanes = 6$ et $busway = lanes$\\
        Troisième cas: $busway = lanes$ mais pas compris dans le nb de $lanes$
    \end{frame}
    \section{Nouvelle fonction de coût}
    \begin{frame}
        \frametitle{Retour sur le coût: calcul du nombre de voies}
        \begin{itemize}
            \item Si on a le nombre de voies
            $$weight = \#lanes$$
            \item Sinon, si on a la width
            $$weight = width \slash 4$$
            \item Sinon:
            \begin{itemize}
                \item si $highway \in \{'primary', 'secondary'\}$ alors $weight = 3$
                \item sinon  $weight = 2$
            \end{itemize}
        \end{itemize}
    \end{frame}

    \section{Robustesse}
    \subsection{Introduction}
    \begin{frame}
        \frametitle{Métriques sur le graphe de base}
        \begin{itemize}
            \item Avg Betweenness Centrality: $0.002820583576670117$
            \item Avg Distance: $0$
            \item Spectral Gap: $0$
            \item Spectral Radius: $0$
            \item Natural Connectivity: $0$
        \end{itemize}
    \end{frame}

    % \begin{frame}
    %     \frametitle{Différence avec les coupes}
    
        
    
    % \end{frame}

    \subsection{Attaque par betweenness}
    \begin{frame}
        \frametitle{Attaque par betweenness: évolution de l'avg bc en 10 itérations}
        \centering
        \includegraphics[scale=0.65]{bc_attack_10.pdf}
    \end{frame}

    \begin{frame}
        \frametitle{Attaque par betweenness: les 10 arêtes}
        \centering
        \includegraphics[scale=0.8]{bc_attack_edges.pdf}
    \end{frame}

    \subsection{Attaque par fréquence de coupe}
    \begin{frame}
        \frametitle{Attaque par fréquence de coupe: les 10 arêtes}
        \centering
        \includegraphics[scale=0.8]{edge_attack_edges.pdf}
    \end{frame}

    \section{composantes connexes}
    \begin{frame}
        \frametitle{Distribution des tailles des composantes connexes}
        \centering
        \includegraphics[scale=0.63]{freq_cc_size.pdf}
    \end{frame}

\end{document}