\documentclass[aspectratio=169]{beamer}
\usetheme{metropolis}

\usepackage{roboto}
\usepackage{mathtools}
\usepackage{fixmath}
\usepackage{graphicx}
\usepackage{tikz}
\usepackage{stmaryrd}
\usepackage{svg}
\usepackage{pdfpages}
\usepackage{algpseudocode}
\usepackage{algorithm}
\usepackage{multicol}

\graphicspath{ {../images/} }
\setbeamertemplate{navigation symbols}{}
\setbeamercolor{background canvas}{bg=}

\newcommand{\N}{\mathbb{N}}
\DeclarePairedDelimiter\Brackets{\llbracket}{\rrbracket}

\title{Nouvelle fonction de coût, coupes contraintes et robustesse, composantes connexes en fonction du déséquilibre}
\subtitle{Stage Casser des Graphes}
\author{Louis Milhaud}
\institute{Complex Networks - LIP6}
\date{\today}

\begin{document}

    \AtBeginSection[]{
    \begin{frame}
    \vfill
    \centering
    \begin{beamercolorbox}[sep=8pt,center,shadow=true,rounded=true]{title}
        \usebeamerfont{title}\insertsectionhead\par%
    \end{beamercolorbox}
    \vfill
    \end{frame}
    }
    % frame 1
    \begin{frame}
        \titlepage
    \end{frame}

    % frame 2
    \begin{frame}
        \frametitle{Outline}
        \tableofcontents
    \end{frame}

    \section{Nouvelle fonction de coût}
    \begin{frame}
        \frametitle{Retour sur le coût: calcul du nombre de voies}
        \begin{itemize}
            \item Si on a le nombre de voies
            $$weight = \#lanes$$
            \item Sinon, si on a la width
            $$weight = width \slash 4$$
            \item Sinon:
            $$weight = 2$$
        \end{itemize}
        Améliorations dans le prétraitement: fusion des voies == addition du nombre de voies
    \end{frame}

\end{document}