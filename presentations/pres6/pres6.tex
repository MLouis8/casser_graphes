\documentclass[aspectratio=169]{beamer}
\usetheme{metropolis}

\usepackage{roboto}
\usepackage{mathtools}
\usepackage{fixmath}
\usepackage{graphicx}
\usepackage{tikz}
\usepackage{stmaryrd}
\usepackage{svg}
\usepackage{pdfpages}
\usepackage{algpseudocode}
\usepackage{algorithm}
\usepackage{multicol}

\graphicspath{ {../images/nocost/} }
\setbeamertemplate{navigation symbols}{}
\setbeamercolor{background canvas}{bg=}

\newcommand{\N}{\mathbb{N}}
\DeclarePairedDelimiter\Brackets{\llbracket}{\rrbracket}

\title{Analyses de Robustesse: Graphe non valué}
\subtitle{Casser des Graphes}
\author{Louis Milhaud}
\institute{Complex Networks - LIP6}
\date{\today}

\begin{document}

    \AtBeginSection[]{
    \begin{frame}
    \vfill
    \centering
    \begin{beamercolorbox}[sep=8pt,center,shadow=true,rounded=true]{title}
        \usebeamerfont{title}\insertsectionhead\par%
    \end{beamercolorbox}
    \vfill
    \end{frame}
    }
    % frame 1
    \begin{frame}
        \titlepage
    \end{frame}

    % frame 2
    \begin{frame}
        \frametitle{Outline}
        \tableofcontents
    \end{frame}

    \section{Quelques scores et distributions}

    \begin{frame}
        \frametitle{Score: moyenne egde Betweenness Centrality}
        \centering
        \includegraphics[scale=0.6]{others/avg_bc_scores_nocost.pdf}
    \end{frame}

    \begin{frame}
        \frametitle{Score: plus grande composante connexe}
        \centering
        \includegraphics[scale=0.6]{others/cc_scores_nocost.pdf}
    \end{frame}

    \begin{frame}
        \frametitle{Score: plus grande composante fortement connexe}
        \centering
        \includegraphics[scale=0.6]{others/biggest_scc_nocost.pdf}
    \end{frame}

    \begin{frame}
        \frametitle{Ordre eBC: changements de distributions après 10 arêtes enlevées}
        \centering
        \includegraphics[scale=0.65]{others/nocost_bcdistr_0-10_bc.pdf}    
    \end{frame}

    \begin{frame}
        \frametitle{Ordre deg: changements de distributions après 10 arêtes enlevées}
        \centering
        \includegraphics[scale=0.65]{others/nocost_bcdistr_0-10_deg.pdf}    
    \end{frame}

    \begin{frame}
        \frametitle{Ordre freq: changements de distributions après 10 arêtes enlevées}
        \centering
        \includegraphics[scale=0.65]{others/nocost_bcdistr_0-10_freq.pdf}    
    \end{frame}

    \begin{frame}
        \frametitle{Visuel des eBC de base}
        \centering
        \includegraphics[trim=0 0 0 7cm, scale=0.7]{others/visubc_0.pdf}
    \end{frame}

    \section{Ordre eBC}

    \begin{frame}
        \frametitle{Des influences locales (1)}
        \includegraphics[trim=0 0 0 7cm, scale=0.7]{bc/visubc_0-1_bc.pdf}
    \end{frame}

    \begin{frame}
        \frametitle{Des influences locales (2)}
        \includegraphics[trim=0 0 0 7cm, scale=0.7]{bc/visubc_0-2_bc.pdf}
    \end{frame}

    \begin{frame}
        \centering
        \frametitle{intéret de la eBC itérative: chaque attaque influence les prochaines (1)}
        \includegraphics[trim=0 0 0 7cm, scale=0.7]{bc/visubc_1_bc.pdf}
    \end{frame}

    \begin{frame}
        \centering
        \frametitle{intéret de la eBC itérative: chaque attaque influence les prochaines (2)}
        \includegraphics[trim=0 0 0 7cm, scale=0.7]{bc/visubc_2_bc.pdf}
    \end{frame}

    \begin{frame}
        \centering
        \frametitle{intéret de la eBC itérative: chaque attaque influence les prochaines (3)}
        \includegraphics[trim=0 0 0 7cm, scale=0.7]{bc/visubc_3_bc.pdf}
    \end{frame}

    \begin{frame}
        \centering
        \frametitle{intéret de la eBC itérative: chaque attaque influence les prochaines (1)}
        \includegraphics[trim=0 0 0 7cm, scale=0.7]{bc/visubc_0-2_bc.pdf}
    \end{frame}

    \begin{frame}
        \centering
        \frametitle{intéret de la eBC itérative: chaque attaque influence les prochaines (2)}
        \includegraphics[trim=0 0 0 7cm, scale=0.7]{bc/visubc_2-3_bc.pdf}
    \end{frame}

    \begin{frame}
        \frametitle{Contrairement a une attaque qui ne s'adapterai pas}
        \centering
        \includegraphics[trim=0 0 0 7cm,scale=0.7]{others/noniterativeBC.pdf}
    \end{frame}

    \begin{frame}
        \frametitle{eBC: un ordre efficace... quand on mesure en eBC}
        \includegraphics[trim=0 0 0 7cm,scale=0.7]{deg/visubc_0-10_deg.pdf}
    \end{frame}

    \begin{frame}
        \centering
        \frametitle{Des perturbations qui s'aggrave (1)}
        \includegraphics[trim=0 0 0 7cm, scale=0.7]{bc/visubc_0-5_bc.pdf}
    \end{frame}

    \begin{frame}
        \centering
        \frametitle{Des perturbations qui s'aggrave (2)}
        \includegraphics[trim=0 0 0 7cm, scale=0.7]{bc/visubc_0-10_bc.pdf}
    \end{frame}

    \section{Ordre freq}

    \begin{frame}
        \frametitle{La fréquence: des pertubations locales assez faibles}
        \includegraphics[trim=0 0 0 7cm, scale=0.7]{freq/visubc_0-1_freq.pdf}
    \end{frame}

    \begin{frame}
        \frametitle{La fréquence: un indicateur biaisé mais apparition d'une stratégie de coupe}
        \includegraphics[trim=0 0 0 7cm, scale=0.7]{freq/visubc_0-10_freq.pdf}
    \end{frame}

    \section{Ordre eBC dans une coupe}
    
    \begin{frame}
        \frametitle{Type de coupe majoritaire}
        \includegraphics[trim=0 0 0 7cm, scale=0.7]{cut/visubc_0-10_cut.pdf}
    \end{frame}

    \begin{frame}
        \frametitle{Evolutions des distributions des eBC}
        \centering
        \includegraphics[scale=0.6]{others/nocostcut_bcdistr_0-10_bc.pdf}
    \end{frame}
    
    \begin{frame}
        \frametitle{Comparaison des distributions: cut (eBC) vs freq}
        \centering
        \includegraphics[scale=0.6]{others/nocostcut_bcdistr_cut-freq.pdf}
    \end{frame}

    \begin{frame}
        \frametitle{Comparaison des scores: average eBC}
        \centering
        \includegraphics[scale=0.6]{others/avgbc_cutvsfreq_nocost.pdf}
    \end{frame}
\end{document}