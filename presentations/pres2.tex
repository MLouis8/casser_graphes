\documentclass[aspectratio=169]{beamer}
\usetheme{metropolis}

\usepackage{roboto}
\usepackage{mathtools}
\usepackage{fixmath}
\usepackage{graphicx}
\usepackage{tikz}
\usepackage{stmaryrd}
\usepackage{svg}
\usepackage{pdfpages}

\graphicspath{ {./images/} }
\setbeamertemplate{navigation symbols}{}
\setbeamercolor{background canvas}{bg=}

\newcommand{\N}{\mathbb{N}}
\DeclarePairedDelimiter\Brackets{\llbracket}{\rrbracket}

\title{Classification des coupes, premier regard sur la betweenness}
\subtitle{Stage Casser des Graphes}
\author{Louis Milhaud}
\institute{Complex Networks - LIP6}
\date{\today}

\begin{document}

    \AtBeginSection[]{
    \begin{frame}
    \vfill
    \centering
    \begin{beamercolorbox}[sep=8pt,center,shadow=true,rounded=true]{title}
        \usebeamerfont{title}\insertsectionhead\par%
    \end{beamercolorbox}
    \vfill
    \end{frame}
    }
    % frame 1
    \begin{frame}
        \titlepage
    \end{frame}

    % frame 2
    \begin{frame}
        \frametitle{Outline}
        \tableofcontents
    \end{frame}

    % frame 3
    \section{Classification des coupes}

    \subsection{analyse des fréquences}
    \begin{frame}{20 arêtes les plus coupées avec $\epsilon = 0.01$}
        \includepdf[pages=-]{20_most_frequent001.pdf}
    \end{frame}

    \begin{frame}{20 arêtes les plus coupées avec $\epsilon = 0.03$}
        \includepdf[pages=-]{20_most_frequent003.pdf}
    \end{frame}

    \begin{frame}{20 arêtes les plus coupées avec $\epsilon = 0.1$}
        \includepdf[pages=-]{20_most_frequent01.pdf}
    \end{frame}
    \section{Betweenness vs fréquence}

    \begin{frame}{Corrélation des deux mesures}
    \begin{itemize}
        \item $\epsilon = 0.01$: Correlation: $0.1$
        \item $\epsilon = 0.03$: Correlation: $0.06$
        \item $\epsilon = 0.1$:  Correlation: $0.03$
    \end{itemize}
    \end{frame}
\end{document}

