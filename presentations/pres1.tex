\documentclass[aspectratio=169]{beamer}
\usetheme{metropolis}

\usepackage{roboto}
\usepackage{mathtools}
\usepackage{fixmath}
\usepackage{graphicx}
\usepackage{tikz}
\usepackage{stmaryrd}

\graphicspath{ {./images/} }
\setbeamertemplate{navigation symbols}{}

\newcommand{\cntxt}{\ \mathrm{context}}
\newcommand{\typ}{\ \mathrm{type}}
\newcommand{\N}{\mathbb{N}}
\newcommand{\app}[2]{App_{[x:\sigma]\tau}(#1, #2)}
\newcommand{\pair}[2]{Pair_{[x:\sigma]\tau}(#1, #2)}
\newcommand{\R}[2]{\mathbold{R}_{[z:\Sigma x:\sigma.\tau]\rho}^{\Sigma}(#1, #2)}
\newcommand{\RN}[3]{\mathbold{R}_{[n: \N]\sigma}^{\N}(#1, #2, #3)}
\newcommand{\C}{Comp}
\newcommand{\Intro}{Intro}
\newcommand{\F}{Form}
\newcommand{\E}{Elim}
\newcommand\defeq{\stackrel{\mathclap{\normalfont\mbox{def}}}{\ =\ }}
\newcommand{\soundness}{\stackrel{\mathclap{\normalfont\mbox{$s$}}}{\ \implies\ }}
\newcommand{\Id}[2]{Id_\sigma(#1,#2)}
\newcommand{\Refl}[1]{Refl_\sigma(#1)}
\newcommand{\RID}[4]{\mathbold{R}_{[x:\sigma,y:\sigma,p:Id_\sigma(x,y)]\tau}^{Id}(#1, #2, #3, #4)}
\newcommand{\bn}{[n:\N,x:\sigma]}
\newcommand{\bi}{[z:\sigma]}
\newcommand{\Pich}{\hat{\Pi}}
\newcommand{\Appch}[2]{\hat{App}_{[x:\sigma]El(T)}(#1,#2)}
\newcommand{\lambdach}{\hat{\lambda} x:\sigma.M^{El(T)}}
\newcommand{\Gamdash}{\Gamma\vdash}
\newcommand{\Fami}{\mathcal{F}am}
\newcommand{\cate}{\mathcal{C}}
\newcommand{\types}{Ty_{\cate}}
\newcommand{\terms}{Tm_{\cate}}
\newcommand{\defin}[1]{\emph{\underline{#1}}}
\newcommand{\xsigmas}{x_1:\sigma_1, ..., x_n:\sigma_n}
\newcommand{\zthetas}{z_1:\theta_1, ..., z_n:\theta_n}
\newcommand{\pp}{\texttt{p}}
\newcommand{\qq}{\texttt{q}}
\newcommand{\vv}{\texttt{v}}
\newcommand{\extension}{\Gamma.\sigma}
\newcommand{\textension}{\langle f,M\rangle_\sigma}
\newcommand{\Mbar}{\overline{M}}

\DeclarePairedDelimiter\Brackets{\llbracket}{\rrbracket}

\title{Etudes du déséquilibre et premieres analyses de coupes}
\subtitle{Stage Casser des Graphes}
\author{Louis Milhaud}
\institute{Complex Networks - LIP6}
\date{\today}

\begin{document}

    \AtBeginSection[]{
    \begin{frame}
    \vfill
    \centering
    \begin{beamercolorbox}[sep=8pt,center,shadow=true,rounded=true]{title}
        \usebeamerfont{title}\insertsectionhead\par%
    \end{beamercolorbox}
    \vfill
    \end{frame}
    }
    % frame 1
    \begin{frame}
        \titlepage
    \end{frame}

    % frame 2
    \begin{frame}
        \frametitle{Outline}
        \tableofcontents
    \end{frame}

    % frame 3
    \section{Déséquilibre}
    % frame 4
    \begin{frame}{Impact du déséquilibre: pour différentes tailles de coupe}
        % TODO: faire les plots des donnees dans le notebook et les mettre ici
    \end{frame}

    \begin{frame}{Impact du déséquilibre: focus sur la taille 2}
        % TODO: creer le plot et le mettre
    \end{frame}

    \begin{frame}{Conclusions sur le déséquilibre}
        % mettre la visu et conclure
    \end{frame}

    \section{Etude de 1000 coupes}
    \begin{frame}
        \frametitle{1000 coupes, déséquilibre = 0.01}
        % TODO mettre les 9 images
    \end{frame}

    \begin{frame}
        \frametitle{1000 coupes, déséquilibre = 0.01}
        %TODO : stats    
    \end{frame}

    \begin{frame}
        \frametitle{1000 coupes, déséquilibre = 0.03}
        %TODO : stats    
    \end{frame}

    \begin{frame}
        \frametitle{1000 coupes, déséquilibre = 0.1}
        %TODO : stats    
    \end{frame}
    \section{Corrélation coupe - mesures}

    \begin{frame}{Betweenness}
    \end{frame}

\end{document}