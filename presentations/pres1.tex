\documentclass[aspectratio=169]{beamer}
\usetheme{metropolis}

\usepackage{roboto}
\usepackage{mathtools}
\usepackage{fixmath}
\usepackage{graphicx}
\usepackage{tikz}
\usepackage{stmaryrd}
\usepackage{svg}

\graphicspath{ {./images/} }
\setbeamertemplate{navigation symbols}{}

\newcommand{\N}{\mathbb{N}}
\DeclarePairedDelimiter\Brackets{\llbracket}{\rrbracket}

\title{Etudes du déséquilibre et premieres analyses de coupes}
\subtitle{Stage Casser des Graphes}
\author{Louis Milhaud}
\institute{Complex Networks - LIP6}
\date{\today}

\begin{document}

    \AtBeginSection[]{
    \begin{frame}
    \vfill
    \centering
    \begin{beamercolorbox}[sep=8pt,center,shadow=true,rounded=true]{title}
        \usebeamerfont{title}\insertsectionhead\par%
    \end{beamercolorbox}
    \vfill
    \end{frame}
    }
    % frame 1
    \begin{frame}
        \titlepage
    \end{frame}

    % frame 2
    \begin{frame}
        \frametitle{Outline}
        \tableofcontents
    \end{frame}

    % frame 3
    \section{Déséquilibre}
    
    % frame 4
    \begin{frame}{Impact du déséquilibre}
        \begin{center}
            \includesvg[scale=0.8]{desequilibres.svg}
        \end{center}
    \end{frame}

    % frame 5
    \begin{frame}{Impact du déséquilibre: focus sur trois valeurs}
       \begin{center}
           \includesvg[scale=0.6]{3desequilibres1000coupes}
       \end{center}
    \end{frame} 

    % frame 6
    \begin{frame}{Conclusions sur le déséquilibre}
        On observe:
        \begin{itemize}
            \item les déséquilibres faibles font apparaitres des coupes a fort coût (contraintes fortes)
            \item sur plusieurs coupes, le minimum est similaire donc pas de probleme si on prend un déséquilibre faible
            \item Pour le reste, sauf précision on se fixe déséquilibre = 0.03
        \end{itemize}
    \end{frame}

    % frame 7
    \section{Etude de 1000 coupes}
    
    % frame 8
    \begin{frame}
        \frametitle{1000 coupes, déséquilibre = 0.01, convergence}
        \begin{center}
            \includegraphics[scale=0.3]{images/convergence_1000_cuts.png}
        \end{center}
    \end{frame}

    % frame 9
    \begin{frame}
        \frametitle{1000 coupes, déséquilibre = 0.01, quelques stats}
        \emph{Stats sur les coupe}
        \begin{itemize}
            \item Meilleure coupe: 332m
            \item Fréquence de la meilleure coupe: 0.044\%
            \item Pire coupe: 784m
            \item moyenne: 376.5m
            \item std: 36.5m
        \end{itemize}
        \emph{Stats sur les arêtes}
        \begin{itemize}
            \item L'arête la plus coupée (658, 18935) l'a été 517 fois
            \item moyenne: 12.2
            \item écart moyen: 29.4
            \item total de 4374 arêtes coupées sur 46761
        \end{itemize}
    \end{frame}

    % frame 10
    \begin{frame}
        \frametitle{1000 coupes, déséquilibre = 0.03, quelques stats}
        \emph{Stats sur les coupe}
        \begin{itemize}
            \item Meilleure coupe: 332m
            \item Fréquence de la meilleure coupe: 0.0439\%
            \item Pire coupe: 450m
            \item moyenne: 348.4m
            \item std: 21.5m
        \end{itemize}
        \emph{Stats sur les arêtes}
        \begin{itemize}
            \item L'arête la plus coupée (1273, 1274) l'a été 559 fois
            \item moyenne: 30.9
            \item écart moyen: 76.3
            \item total de 1563 arêtes coupées sur 46761
        \end{itemize} 
    \end{frame}

    % frame 11
    \begin{frame}
        \frametitle{1000 coupes, déséquilibre = 0.1, quelques stats}
        \emph{Stats sur les coupe}
        \begin{itemize}
            \item Meilleure coupe: 328m
            \item Fréquence de la meilleure coupe: 0.438\%
            \item Pire coupe: 397m
            \item moyenne: 332.2m
            \item std: 8.9m
        \end{itemize}
        \emph{Stats sur les arêtes}
        \begin{itemize}
            \item L'arête la plus coupée (1273, 1274) l'a été 532 fois
            \item moyenne: 61.8
            \item écart moyen: 146.3
            \item total de 747 arêtes coupées sur 46761
        \end{itemize}
    \end{frame}

    % frame 12
    \subsection{Visualisation}
    \begin{frame}{Visualisation des arêtes les plus coupées}
        \begin{center}
            \includegraphics[]{images/special_edges.png}
        \end{center}
    \end{frame}

    % frame 13
    \begin{frame}{Visualisation des fréquences}
    \begin{center}
        \includegraphics[scale=0.95]{images/Paris_frequency.png}
    \end{center}
    (400: marron, 300: violet, 200: rouge, 100: orange, 50: vert)
    \end{frame}

    % frame 14
    \section{Corrélation coupe - mesures}

    % frame 13
    \begin{frame}{Corrélation coupe - mesures}
    \emph{Mesure: correlation}
    \begin{itemize}
        \item betweenness:  
    \end{itemize}
    \end{frame}

\end{document}
